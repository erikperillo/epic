\documentclass[11pt]{article}
\usepackage{graphicx}
\usepackage{float}
\usepackage{amsmath}
\usepackage{amsfonts}
\usepackage[brazilian]{babel}
\usepackage[utf8]{inputenc}
\usepackage[backend=biber]{biblatex}
\usepackage{csquotes}
%\usepackage{docmute}
\usepackage{array}
\ifdefined\multicol
	\usepackage{multicol}
	\usepackage{geometry}
\fi
\usepackage[T1]{fontenc}
\addbibresource{rel_parcial_erik_perillo_2sem2016.bib}

\newcommand{\fromeng}[1]{\footnote{do inglês: \textit{#1}}}
\newcommand{\tit}[1]{\textit{#1}}
\newcommand{\tbf}[1]{\textbf{#1}}
\newcommand{\ttt}[1]{\texttt{#1}}

\begin{document}

\begin{titlepage}
	\centering
	{\scshape\Large Relatório Parcial\par}
	\vspace{1.5cm}
	{\huge\bfseries Processos atencionais e aprendizado de máquina
		para sistemas robóticos\par}
	\vspace{1cm}
	{\itshape Aluno: Erik de Godoy Perillo\par}
	{\itshape Orientadora: Profa. Dra. Esther Luna Colombini\par}
	\vspace{0.5cm}
	\vfill
    Instituto de Computação\\
	Universidade Estadual de Campinas
	\vfill
	{\large \today\par}
\end{titlepage}

\newpage

\section{Introdução}
\subsection{Atenção em sistemas robóticos}
Breves comentários:
como a atenção proporciona um papel fundamental na navegação de robôs.

\subsection{Objetivos do trabalho}
Queremos um framework eficiente para atenção em sistemas robóticos.
O trabalho focará mais em atenção visual.
Os objetivos do primeiro semestre eram: X, Y, Z. Cronograma.


\section{Resumo das atividades}
\subsection{Revisão Bibliográfica}
Revemos as bases para atenção.
Top-down. Bottom-up.
Vimos modelos clássicos: Vocus, esther, robozinho.

\subsection{Formulação de um modelo inicial}
Escolhemos o vocus para nos basear, focando no bottom-up.
Descrição básica das ideias do modelo.

\subsubsection{Implementação: att}
Repositório.
Modelo geral.
Mapas.
Center-surround.
Escolha da região mais saliente.

\subsubsection{Resultados}
Imagens. Resultados intuitivos.

\subsubsection{Comparações}
Métricas.
Comparações com modelos no topo.

\subsection{Modelos novos}
Deep learning everywhere.

\section{Produção Científica}
Modelo Att.
Notas: Estudo de métricas.
Estudo de sistemas com Deep Learning.

\section{Próximos passos}
DeepFix. Vídeo.

\printbibliography

\end{document}
