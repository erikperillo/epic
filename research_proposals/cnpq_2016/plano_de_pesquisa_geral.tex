\documentclass[11pt]{report}
\usepackage{graphicx}
\usepackage{float}
\usepackage{amsmath}
\usepackage{amsfonts}
\usepackage[brazilian]{babel}
\usepackage[utf8]{inputenc}
\usepackage[T1]{fontenc}

\newcommand{\fromeng}[1]{\footnote{do inglês: \textit{#1}}}
\newcommand{\tit}[1]{\textit{#1}}
\newcommand{\tbf}[1]{\textbf{#1}}
\newcommand{\ttt}[1]{\texttt{#1}}

\begin{document}

\begin{titlepage}
	\centering
	{\scshape\Large Projeto de pesquisa\par}
	\vspace{1.5cm}
	{\huge\bfseries Título\par}
	\vspace{2cm}
	{\Large\itshape Aluno\par}
	{\Large\itshape Orientador\par}
	\vfill
	Universidade 
	\vfill
	{\large \today\par}
\end{titlepage}

\newpage

\section{Introdução}

\subsection{Ambientação}
\paragraph{}
Aqui vem a ambientação do problema, isto é, a esfera S de aplicação
a qual inspirou nosso projeto e sua relevância na sociedade.

\subsection{Revisão bibliográfica breve}
\paragraph{}
Aqui se fala sobre as técnicas atuais que servem à esfera S ambientada.

\subsection{Motivação}
\paragraph{}
Aqui se introduz o(s) problema(s) P existente(s) na esfera S sobre o qual 
queremos trabalhar. 
Comentamos que melhorar P é relevante para S por um motivo M.
Descrevemos como as técnicas atuais têm X, Y e Z como defeitos e
melhorá-los de forma F trará benefícios B a P porque R.

\subsection{Introdução à nossa abordagem}
\paragraph{}
Aqui há uma descrição breve das técnicas que pretendemos usar.

\subsection{Proposta}
\paragraph{}
Deve-se comparar nossa abordagem com as outras discutidas e salientar suas
diferenças para mostrar originalidade.
Deve-se então explicar por que essas diferenças podem trazer os benefícios 
B esperados.

\section{Objetivos}
\paragraph{}
Aqui colocamos o objetivo principal da pesquisa. Devemos buscar algo que
seja de utilidade para a sociedade de alguma forma, seja geral e original.

\section{Métodos}
\paragraph{}
Aqui falamos sobre os meios usados para se chegar aos objetivos.

\section{Cronograma}
\paragraph{}
Aqui descrevemos as atividades planejadas no decorrer do tempo.

\section{Bibliografia}
\paragraph{}
Auto-explanatório.

\end{document}
