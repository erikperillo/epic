\documentclass[11pt]{article}
\usepackage{graphicx}
\usepackage{float}
\usepackage{amsmath}
\usepackage{amsfonts}
\usepackage[brazilian]{babel}
\usepackage[utf8]{inputenc}
\usepackage[T1]{fontenc}

\newcommand{\fromeng}[1]{\footnote{do inglês: \textit{#1}}}
\newcommand{\tit}[1]{\textit{#1}}
\newcommand{\tbf}[1]{\textbf{#1}}
\newcommand{\ttt}[1]{\texttt{#1}}

\begin{document}

\begin{titlepage}
	\centering
	{\scshape\Large Projeto de pesquisa\par}
	\vspace{1.5cm}
	{\huge\bfseries Processos atencionais e aprendizado de máquina 
		para sistemas robóticos\par}
	\vspace{2cm}
	{\Large\itshape Aluno: Erik de Godoy Perillo\par}
	{\Large\itshape Orientadora: Esther Luna Colombini\par}
	\vfill
	Universidade Estadual de Campinas 
	\vfill
	{\large \today\par}
\end{titlepage}

\newpage

\section{Introdução}

\subsection{Ambientação}
\paragraph{}
\tbf{Aqui vem a ambientação do problema, isto é, a esfera S de aplicação
a qual inspirou nosso projeto e sua relevância na sociedade.}

\tit{Nosso caso}: 
\begin{itemize}
	\item Falar sobre sistemas robóticos em geral.
	\item Falar sobre a tendência de sistemas robóticos autônomos ser uma 
		área quente.
	\item Discorrer sobre a navegação em sistemas robóticos.
\end{itemize}

\subsection{Revisão bibliográfica breve}
\paragraph{}
\tbf{Aqui se fala sobre as técnicas atuais que servem à esfera S ambientada.}

\tit{Nosso caso}:
\begin{itemize}
	\item Comentar sobre técnicas que possibilitam a navegação robótica 
		atualmente.
\end{itemize}

\subsection{Motivação}
\paragraph{}
\tbf{Aqui se introduz o(s) problema(s) P existente(s) na esfera S sobre o qual 
queremos trabalhar. 
Comentamos que melhorar P é relevante para S por um motivo M.
Descrevemos como as técnicas atuais têm X, Y e Z como defeitos e
melhorá-los de forma F trará benefícios B a P porque R.}

\tit{Nosso caso}:
\begin{itemize}
	\item O volume de dados manipulado pelos sistemas robóticos autônomos é
		em geral demasiadamente grande e muita da informação nele contida
		não é relevante para o problema em questão. Ao mesmo tempo, robôs
		autônomos em geral são muitas vezes limitados pelo seu hardware 
		embarcado que tende a ter poder de processamento relativamente baixo.
	\item Obter técnicas que permitam uma navegação mais sofisticada
		é importante porque isso daria mais poder aos sistemas 
		autônomos que temos hoje para cumprir uma gama maior de tarefas.
	\item Usar técnica X é ruim porque não é eficiente. Usar técnica Y é
		ruim porque não é muito geral.
	\item Há uma nova tendência de GPUs embarcadas que de serem agora usadas, 
		o que possibilita processamento pesado para certas operações e 
		consequentemente proporciona uma maior liberdade na escolha de técnicas
		mais poderosas.
	\item Busca-se então por técnicas mais sofisticadas, tendo-se o apoio das
		GPUs embarcadas como um fato, com a preocupação da eficiência para
		sistemas em tempo real.
\end{itemize}

\subsection{Introdução à nossa abordagem}
\paragraph{}
\tbf{Aqui há uma descrição breve das técnicas que pretendemos usar.}
\begin{itemize}
	\item Falar sobre processos atencionais.
	\item Falar sobre técnicas de reconhecimento de padrões em Machine Learning.
\end{itemize}

\tit{Nosso caso}:
Processos atencionais, aprendizado de máquina para identificação de objetos...

\subsection{Proposta}
\paragraph{}
\tbf{Deve-se comparar nossa abordagem com as outras discutidas e salientar suas
diferenças para mostrar originalidade.
Deve-se então explicar por que essas diferenças podem trazer os benefícios 
B esperados.}

\tit{Nosso caso}:
Mesclar processos atencionais com ML agora seria  
seria eficiente computacionalmente e geral/robusto. 
Isso nos leva a buscar uma alternativa que faça uso dessas técnicas.

\section{Objetivos}
\paragraph{}
\tbf{Aqui colocamos o objetivo principal da pesquisa. Devemos buscar algo que
seja de utilidade para a sociedade de alguma forma, seja geral e original.}

\tit{Nosso caso}:
Obter técnicas que permitam a navegação de sistemas robóticos autônomos 
em geral.

Fazer uma avaliação da eficiência de nossa técnica.

Citar um framework para GPUs embarcadas que desejamos ter ao final para 
treinamento, classificação etc.

\section{Métodos}
\paragraph{}
\tbf{Aqui falamos sobre os meios usados para se chegar aos objetivos.}

\tit{Nosso caso}:
Falar sobre a plataforma de testes. 
Citar o projeto Piranha aqui. 
Falar do uso da câmera. Falar de GPUs aqui. 
Falar sobre os tipos de testes para avaliar a robustez, eficiência das técnicas.

\section{Cronograma}
\paragraph{}
\tbf{Aqui descrevemos as atividades planejadas no decorrer do tempo.}

\bibliographystyle{plain}
\bibliography{plano_de_pesquisa}

\end{document}
