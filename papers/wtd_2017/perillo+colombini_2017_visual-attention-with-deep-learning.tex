% -- NAO MUDAR O TAMANHO DA FONTE.
% -- O USO DO TAMANHO DA FONTE EM 12PT Eh OBRIGATORIO
\documentclass[12pt]{article}

% -- O USO DO TEMPLATE DA SBC Eh OBRIGATORIO
% -- NAO ALTERAR NENHUMA PARAMETRO DO MODELO DA SBC
% -- (E.G., MARGENS LATERAIS, RODAPE, CABECA�HO, ENTRE OUTROS)
\usepackage{sbc-template}

\usepackage{graphicx,url}

%\usepackage[brazil]{babel}
\usepackage[latin1]{inputenc}

% descomente abaixo caso tenha problemas com a numera��o das se��es n�o
% aparecerem. Em especial no Ububtu 16.x
%\usepackage{etoolbox}
%\makeatletter
%\patchcmd{\ttlh@hang}{\parindent\z@}{\parindent\z@\leavevmode}{}{}
%\patchcmd{\ttlh@hang}{\noindent}{}{}{}
%\makeatother

%macros
\newcommand{\tit}[1]{\textit{#1}}
\newcommand{\tbf}[1]{\textbf{#1}}
\newcommand{\ttt}[1]{\texttt{#1}}

% -- O USO DESTE COMANDO Eh OBRIGATORIO
\sloppy

% -- CERTIFIQUE-SE DE QUE O TITULO NAO ESTEJA ULTRAPASSANDO OS LIMITES
% -- DAS MARGENS.
% -- USE O COMANDO QUEBRA-DE-LINHA (\\) SE TITULO FOR MUITO LONGO.
\title{Visual Attention with Deep Learning}

% -- CERTIFIQUE-SE DE QUE O NOME DOS AUTORES ESTEJA CORRETO.
% -- EM GERAL COLOCA-SE O ALUNO COMO PRIMEIRO AUTOR E O ORIENTADOR COMO
% -- ULTIMO AUTOR.
\author{Erik Perillo\inst{1}, Esther Luna Colombini\inst{1}}

% -- CERTIFIQUE-SE DE QUE OS NOMES DAS INSTITUICOES E SEUS RESPECTIVOS ENDERECOS
% -- NAO ESTEJAM ULTRAPASSANDO OS LIMITES DAS MARGENS. SE ISTO OCORRER, USE O
% -- COMANDO QUEBRA-DE-LINHA (\\) OU ABREVIACOES APROPRIADAS.
\address{Institute of Computing (IC) -- University of Campinas
  (Unicamp)\\
  Caixa Postal 6176 -- 13.084-971 -- Campinas -- SP -- Brazil
  \email{erik.perillo@gmail.com, esther@ic.unicamp.br}
}


% -- CERTIFIQUE-SE DE QUE AS TABELAS E FIGURAS INSERIDAS NO
% -- DOCUMENTO NAO ESTEJAM ULTRAPASSANDO
% -- OS LIMITES DAS MARGENS.

\begin{document}

\maketitle

% -- O RESUMO EM INGLES Eh OBRIGATORIO,
% -- INDEPENDENTE DO IDIOMA EM QUE TRABALHO FOI ESCRITO
\begin{abstract}
Vision is a key element in one's process of understanding the world.
The high volume of sensorial data is however problematic because
most of the information is often irrelevant.
Humans realize sensorial filtering by what we call attention.
We propose the application of Deep Learning for obtaining a visual salience
system which behaves similarly to humans.
We built a new convolutional neural network with relatively
simple architecture, yielding a performance level consistently among the
best ten state of the art models in MIT300 benchmark.
\end{abstract}

% -- O RESUMO EM PORTUGUES Eh OBRIGATORIO, INDEPENDENTE DO
% -- INDIOMA EM QUE O TRABALHO FOI ESCRITO
\begin{resumo}
A vis�o � elemento-chave no processo de entender o mundo para um ser.
Entretanto, a alta quantidade de dados sensoriais � problem�tica,
havendo muitas vezes irrelev�ncia de informa��o.
Nos seres humanos, h� um filtro sensorial realizado pela aten��o.
Propomos a aplica��o de
\tit{Deep Learning} para a obten��o de um sistema de sali�ncia visual que
se comporte como o dos seres humanos.
Constru�mos uma nova rede neural convolucional de arquitetura relativamente
simples e com um desempenho que a coloca consistentemente entre os dez
melhores modelos estado da arte no \tit{MIT300 benchmark}.
\end{resumo}

\section{Introduction}
\begin{itemize}
	\item What's attention?
	\item Visual saliency
    \item Old models, new methods
\end{itemize}
\section{Objectives}
\begin{itemize}
    \item Use old techniques to new, good model
    \item Try to be simple
\end{itemize}
\section{Methodology}
\begin{itemize}
    \item Old method: be succint
    \item New method: architecture, datasets, datapreproc, training
\end{itemize}
\section{Results}
\begin{itemize}
    \item Old method
    \item New method, mit300 bm
\end{itemize}
\section{Conclusion}
\begin{itemize}
    \item Ey, that's pretty good
    \item next steps
\end{itemize}

% -- NAO MUDAR O ESTILO DAS REFERENCIAS BIBLIOGRAFICAS. O USO DO PADRAO DA SBC Eh OBRIGATORIO
\bibliographystyle{sbc}
\bibliography{sbc-template}

\end{document}
